\chapter{D�cret no 95-562 du 6 mai 1995 (articles 1\ier{} et 2\ieme)}
\label{decret}


D�cret \no95-562 du 6 mai 1995 relatif aux centres communaux et intercommunaux d'action sociale ainsi qu'aux sections de centre communal d'action sociale des communes associ�es et portant dispositions particuli�res applicables aux centres communaux d'action sociale de Marseille et de Lyon :

\begin{center}
	\vspace*{.5cm}
	\Large{CHAPITRE I\ier{} -- Dispositions g�n�rales}
\end{center}


\vspace*{.5cm}

\subparagraph*{Art. 1\ier.}
-- Les centres communaux et intercommunaux d'action sociale mentionn�s au chapitre II du titre III du code de la famille et de l'aide sociale proc�dent annuellement � une analyse des besoins sociaux de l'ensemble de la population qui rel�ve d'eux, et notamment de ceux des familles, des jeunes, des personnes �g�es, des personnes handicap�es et des personnes en difficult�.
Cette analyse fait l'objet d'un rapport pr�sent� au conseil d'administration.

\subparagraph*{Art. 2.}
-- Les centres d'action sociale mettent en oeuvre, sur la base du rapport mentionn� � l'article 1er, une action sociale g�n�rale, telle qu'elle est d�finie par l'article 137 du code de la famille et de l'aide sociale et des actions sp�cifiques.
Ils peuvent intervenir au moyen de prestations en esp�ces, remboursables ou non, et de prestations en nature.

\begin{center}
	\vspace*{1cm}
	Texte complet disponible sur Legifrance : \\ 
	\textit{\href{http://www.legifrance.gouv.fr/affichTexte.do?cidTexte=JORFTEXT000000187191&fastPos=2&fastReqId=298048010&categorieLien=id&oldAction=rechTexte}{http://www.legifrance.gouv.fr/affichTexte.do?cidTexte=JORFTEXT000000187191}}
\end{center}